\documentclass[12pt]{report}

\usepackage{parskip}
  
\begin{document}

1. For each of the following subproblems, 
there are two functions being compared. 
Call the first $f(n)$ and the second $g(n)$. 
Prove whether $f(n)$ is $O(g(n))$ or not, 
and also whether it is $\Omega(g(n))$ or not.
Note that the requirement is to prove, not just state.

\textbf{(a)} $3^{n+1}$ vs $3^n$

Let us assume that $3^{n+1}$ is $O(3^n)$ for all $n \geq 1$.

Therefore some constant $c$ must exist $c < \infty$ such that $3^{n+1} \leq c * 3^n$, meaning that $3 \leq c$.

The constants that make this statement true are $c \geq 3$ and $n_0 \geq 1$

Therefore $3^{n+1}$ is $O(3^n)$

Let us assume $3^{n+1}$ is $\Omega(3^n)$.

Therefore some constant $c$ must exist such that $3^{n+1} \geq c * 3^n$.

This means that $c$ must satisfy $3 \geq c$ for all $n$.

Since c can be any constant, we can say that $3^{n+1}$ is $\Omega(3^n)$

\textbf{(b)} $2^{2n}$ vs $2^n$

If we wish to say $2^{2n}$ is $O(2^n)$, there must exist
two constants $c$ and $n_0$ such that $c*2^n \geq 2^{2n}$
for all $n \geq n_0$

The constants that will fulfill this requirement are $c\geq 2$
and $n_0 \geq 1$

Therefore $2^{2n}$ is $O(2^n)$.

Let us assume that $2^{2n}$ is $\Omega(2^n)$.

Therefore some constant $c$ must exist such that $2^{2n} \geq c * 2^n$.

This means that $c$ must satisfy $2^n \geq c$ for all $n \geq n_0$

\pagebreak

\textbf{(c)} $4^n$ vs $2^{2n}$

Let us first simplify $2^{2n}$ to $4^n$

For both $O(f(n))$ and $\Omega(f(n))$, all functions are
big-O and Omega of themselves.

\textbf{(d)} $n^2$ vs $n^{2.01}$

Let us assume that $n^2$ is $O(n^{2.01})$.

That means that some constant $c$ must exist such that $n^2 \leq c * n^{2.01}$

Therefore $c$ must satisfy $1/{n^{0.1}} \leq c$ for all
$n \geq n_0$

Since the function $1/{n^{0.1}}$ is a decreasing function we can
say that $n^2$ is $O(n^{2.01})$.

Let us assume that $n^2$ is $\Omega(n^{2.01})$.

This means that some constant c must exist such that $n^2 \geq c * n^{2.01}$.

Therefore c must satisfy $1/n^0.1 \geq c$ for all $n\geq n_0$

Because $1/n^0.1$ is a decreasing function, there cannot exist a $c$ that will satisfy this function.

Therefore $n^2$ is NOT $\Omega(n^{2.01})$.

\textbf{(e)} $n^{0.9}$ vs $0.9^n$

Let us assume that $n^{0.9}$ is $O(0.9^n)$.

This means that some constant $c$ must exist such that $n^{0.9} \leq c * 0.9^n$.

Therefore $c$ must satisfy $e^{.9ln(n) - nln(.9)} \leq c$

This equation increases exponentially, and so there is no value of $c$ that will satisfy this
for all $n \geq n_0$.

Therefore $n^{0.9}$ is NOT $O(0.9^n)$.

For $\Omega$ the same is true except $c$ must satisfy  $e^{.9ln(n) - nln(.9)} \geq c$ instead.

We know that the equation is increasing at an exorbitant rate, so there definitely exists a
$c$ to satisfy this equation for all $n \geq n_0$.

Therefore $n^{0.9}$ is $\Omega(0.9^n)$.

\pagebreak

2. Same instructions as problem 1.

\textbf{(a)} $\log^c n$ vs $\log n$, where $c$ is a constant greater than 1.

Let us assume that $\log^c n$ is $O(\log n)$.

This means that some constant $k$ must exist such that $log^c n \leq k * \log n$.

Therefore $k$ must satisfy $\log^{c - 1}n \leq k$.

$log n$ is an increasing function, so there exists no value of $k$
that will satisfy this equation for all values of $c$ or $n$.

For $\Omega$, the inequality can be flipped to say that
$k$ must instead satisfy $\log^{c - 1}n \leq k$.

Since $\log^{c-1}n$ is an increasing function, it is easy to find a value of
$k$ that will satisfy this equation for all values of $c$ and $n$.

Therefore this set is only $\Omega$.

\textbf{(b)} $\log n^c$ vs $\log n$, where $c=\Theta(1)$.

For this set of functions the constant $k$ will again be used.

It must satisfy the inequality $c \leq k$ for $O$, and $c \geq k$ for $\Omega$.

Since we can just set $k = c$, these are both satisfied for all values of $c$ and $n$.

Therefore this set of functions is both $O$ and $\Omega$.

\pagebreak

\textbf{(c)} $\log(c \cdot n)$ vs $\log n$, where $c = \Theta(1)$.
You can assume both have the same base.

For this set of functions the const $k$ will again be used.

Let us assume that $\log(c \cdot n)$ is $O(\log n)$.

This means that $k$ must satisfy the equation $\log(c \cdot n) \leq k \cdot \log n$.

Which can be rewritten as $\log_n c - 1 \leq k$.

This function is decreasing, so we can say it is trivial to find a value of k that will
satisfy all values of $c$ and $n \geq n_0$.

We can flip the inequality for $\Omega$, and by the same logic we can say
that there exists no constant that will satisfy the flipped inequality.

Therefore this set is $O$ and not $\Omega$.

\textbf{(d)} $\log_a n$ vs $\log_b n$, where $a$ and $b$ are constants greater
than 1. Show that you understand why this restriction on $a$ and $b$ was given.

The restrictions exist because for logarithms with base $<$ 1, any
multiplication or division with it requires that the inequality be
flipped.

For this equation, we can use $c$ again!

Let us assume that $\log_a n$ is $O(\log_b n)$.

This means that $c$ must satisfy the inequality $\log_a b \leq c$.

For $\Omega$, the inequality is $\log_a b \geq c$.

Therefore we can say that $c$ can equal $\log_a b$, satisfying
both inequalities for all values of $a$, $b$, and $n$ all at once.

This set is both $O$ and $\Omega$.

\pagebreak

3. Let $f(x) = O(x)$ and $g(x) = O(x)$. Let $c$ be a positive constant.
Prove or disprove that $f(x) + c \cdot g(y) = O(x + y)$.

Let us assume that $f(x) + c \cdot g(y)$ is $O(x + y)$.

This means there exists some const $k$ that satisfies the inequality

$f(x) + c\cdot g(y) \leq k \cdot (x + y)$

If $f(x) = O(x)$, then there exists some constant $a$ that satisfies
the inequality $f(x) \leq a \cdot x$.

Same for $g(x)$ and constant $b$.

If we say that $k$ will satisfy \\$k \geq a \cdot b \cdot c$, then
\\$k \cdot x \geq f(x)$ is true, and
\\$k \cdot y \geq c \cdot g(x)$ is true as well, therefore
\\$f(x) + c\cdot g(y) \leq k \cdot (x + y)$ is true as well.

Therefore $f(x) + c \cdot g(y) = O(x+ y)$.

\pagebreak

4. Let $f(n) = \sum_{x=1}^n(\log^3n \cdot x^{29})$. Find a simple $g(n)$ such that
$f(n) = \Theta(g(n))$. (Prove both big-$O$ and $\Omega$). Don't use induction / substitution,
or calculus, or any fancy formulas.

Just exaggerate and simplify.

For a geometric series, the $n$th partial sum is equal to $\frac{n(n+1)}{2}$.

Using this to simplify $f(n)$ to
$\frac{\log^3n \cdot n^{29} \cdot \log^3n \cdot (n + 1)^{29}}{2}$.

Simplify to  $\frac{1}{2} \cdot \log^6n \cdot n^{29} \cdot (n + 1)^{29}$.

We can remove the constant in the beginning and the constant inside $(n + 1)$
to further simplify.

$\log^6n \cdot n^{58}$.

We proved earlier that $\log^cn = \Omega(\log n)$ and
$n^c = \Omega(c^n)$.

Therefore we can further simplify to $\log n \cdot 58^n$.
and $58^n$ is $\Omega(2^n)$, so 

$g(n) = \log n \cdot 2^n$ and $f(n) = \Omega(g(n))$.

We had proved earlier that $\log^cn$ is not $O(\log n)$, so
this does not work for big-$O$.

\end{document}