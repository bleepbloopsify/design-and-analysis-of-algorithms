\documentclass[12pt]{article} 

\usepackage[top=2.54cm, bottom=2.54cm, left=2.54cm, right=2.54cm]{geometry}
\usepackage{enumerate}
\pagestyle{plain}

\begin{document}  
\pagestyle{empty}
 

\begin{center} ALGORITHMS,   FALL 2018, HOMEWORK 7
\end{center}
\noindent  {\bf  Due Sunday, October 21 at 11:59pm.  No late submissions, no extensions.}\\ 

\noindent Worth 1\% of the final grade.\\

\begin{enumerate}

 \item 
Let  $S = \{s_1, \ldots, s_n\}$ be a list of distinct real numbers. 


(a) Show how to find the smallest value $|s_i-s_j|$ $(i\neq j)$ in  $O(n\log n)$ time.   

(b) Now suppose that there will be a mix of operations: besides  needing to handle multiple queries such  as the one in part (a), you must also allow insertions into $S$.  
  Show how to maintain a simple data structure that can answer a query in constant time, and that takes at most logarithmic time to update after each insertion.  \\
  If you can't do constant time queries, then at least manage logarithmic time.  Your data structure should be as simple as possible, in the sense that you should not overload it unnecessary extra information.
    

(c) Assuming you have used the simplest possible data structure in (b), it will probably be hard to also handle deletions from $S$ in logarithmic time.
Show how to handle deletions by storing extra information. \\




\end{enumerate}



\end{document}






















