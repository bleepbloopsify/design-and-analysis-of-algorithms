\documentclass[12pt]{article} 

\usepackage[top=2.54cm, bottom=2.54cm, left=2.54cm, right=2.54cm]{geometry}
\usepackage{enumerate}
\pagestyle{plain}

\usepackage{qtree}

\begin{document}  
\pagestyle{empty}
  

\begin{center} ALGORITHMS,   FALL 2018, HOMEWORK 2
\end{center}
\noindent Due Thursday, September 20 at 11:59pm.  \\
Worth 2\% of the final grade.\\
Submit each problem on a separate page. Subproblems can be on the same page.

\begin{enumerate}

\item
$S(n) = 2S(\frac{n}{2})  + \Theta(1)$.  

(a) Evaluate $S(n)$ with a recursion tree. 

\Tree
[.$S(\frac{n}{2})$ 
  [.$S(\frac{n}{4})$
    [.$S(\frac{n}{8})$ ]
    [.$S(\frac{n}{8})$ ]
  ]
  [.$S(\frac{n}{4})$
    [.$S(\frac{n}{8})$ ]
    [.$S(\frac{n}{8})$ ]
  ]
]

Each level of this tree costs $\Theta(1)$, and there are $\log_2 n$ levels.

$S(1) = \Theta(1) = c_2$

$S(n) = c_2n + \log_2 n$

$S(n) = \Theta(n)$


(b) Use substitution (induction) to get a lower bound that matches the result in (a).    

$S(n) = 2 \cdot S(\frac{n}{2}) + 1$

Assume that $S(n) \leq d \cdot n$.

Therefore for all $k < n$, $S(k) \leq d\cdot k$

Substitute: $S(n) \leq 2 \cdot d \frac{n}{2} + c \cdot 1$

$ = d n + 1$

$ = d  n + 1$

$ \leq d n$ if $d \geq c$

(c)  {\em Not required or graded:    confirm the  matching upper bound via substitution.}\\

{\em Skipped}

\pagebreak

\item (a) Use  substitution (induction)  to prove: $T(n) = 18T(\frac{n}{3}) + \Theta(n^2)  = O(n^3)$.   

$T(n) = 18T(\frac{n}{3}) + c \cdot n^2$

Assume that $T(n) \leq 18 \cdot c \cdot (\frac{n}{3} )^2 + d \cdot n^2$

Therefore for all $k < n$, $T(k) \leq d \cdot k^3$

Substitute $c \cdot n^3 + dn^2$

Therefore it is leaf dominated with a runtime of $\Theta(n^3)$.
  
(b) Show that this isn't the best possible upper bound for $T(n)$. 

Subsitute: $T(n) \leq  c \cdot 2 \cdot n^2 + d n^2$

$\leq n^2$.

(c) {\em  Not required or graded: confirm (b) by getting a better bound via substitution.}\\

\pagebreak


\item Use the master method for the following, or explain why it's not possible.  If you get Case 3 you  do not need to confirm that there is a geometric series.
\begin{enumerate}

\itemsep-0.35cm

\item  $T(n)= 10\cdot T(\frac{n}{3}) + \Theta(n^2\log^5 n)$.\\

Using master method, we have to compare $n^{\log_310}$ and $n^2\log^5n$\\
$n^{\log_310}$ is slightly larger, and it represents the leaves, so this is leaf-dominated.\\
Therefore the runtime is $\Theta(n^{\log_310})$.\\

\item $T(n) = T(\frac{19n}{72}) + \Theta(n^2)$.\\

Using master method, we have to compare $n^0 = 1$ and $n^2$.\\
$n^2$ is the clear winner here, representing the root cost, so this is root-dominated.\\
Therefore the runtime is $\Theta(n^2)$.\\

\item $T(n) = n\cdot T(\frac{n}{5}) + n^{\log_5 n}$.\\

Using master method, we have to compare $n^{\log_5 n}$ and $n^{\log_5n}$.\\
Because they're the same, the runtime is $n^{\log_5n}\log n$.\\

\item  $T(n) =  3\cdot T(\frac{n}{2}) + n^2$.\\

Using master method, we have to compare $n^{\log_23}$ and $n^2$.\\
$n^2$ is larger, so this function is root dominated and so the runtime is $\Theta(n^2)$.\\
  
\item $T(n) =   T(\frac{n}{n{-}1}) + 1$.\\

The master method does not work here because the value of $b$ derived from this function would be less than 1.\\
  
\item  $T(n) =  4\cdot T(\frac{n}{16}) + \sqrt{n}\cdot\log^4n$. \\

Using the master method we have to compare $n^{\log_{16}4}$ and $\sqrt{n}\cdot\log^4n$.

$\sqrt{n}\cdot\log^4n$ is slightly larger, so this is root-dominated and the runtime is $\Theta(\sqrt{n}\cdot\log^4n)$.
\end{enumerate}

\pagebreak
  
  
\item  Solve $T(n) = T(\sqrt{n}) + \log n$\\
(a) with a recursion tree       \\



In this tree there are $\log n$ levels, and on each level we do $\log n$ work, so we have a runtime of $\log(\log n)$.

(b) by substitution (induction)\\

$T(n) \geq c \cdot \log n$.

For all $k < n, T(k) \geq c \cdot \log k$.

$T(n) \geq c \cdot \log(\sqrt n) + \log n$.

$= \frac{1}{2} \cdot c \cdot \log n + \log n$.

Remnants are $c \cdot \log n$.

Therefore this function is $\Theta(\log n)$.

(c) with the master method, after applying a change of variables, $n=2^m$. 

$T(2^m) = T(2^{\frac{m}{2}}) + \log 2^m$

$f(n) = \log 2^m$

If we were to take the entire set of elements as our parameter, and said that $m=n^2$.

Then our equation would  equal $S(m) = 4S(\frac{m}{4}) + m$, which is just $\Theta(m)$. 

It is similar in the case of $n^2$, where the runtime would be $\Theta(n^2)$.

Similarly in this one, the runtime just becomes $\Theta(m)$.

\end{enumerate}



\end{document}
  
  
  
  
  
  
  
  
  
  
  
  
  
  
  
  
  
  
  
  
  
  
  