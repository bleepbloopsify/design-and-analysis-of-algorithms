\documentclass[12pt]{article} 

\usepackage[top=2.54cm, bottom=2.54cm, left=2.54cm, right=2.54cm]{geometry}
\usepackage{enumerate}
\pagestyle{plain}

\begin{document}  
\pagestyle{empty}
 

\begin{center} ALGORITHMS,   FALL 2018, HOMEWORK 4
\end{center}
\noindent Due Thursday, October 4 at 11:59pm.  \\
Worth 2\% of the final grade.\\

\begin{enumerate}
 \item 
(a) Given a list  of $n$ real numbers, show how to decide in linear time whether it contains at least  $\lceil \frac{n}{2} \rceil$ numbers, all with equal value.  

You iterate over the list, keeping track of each number and the number of appearances it has, and then iterate over the list of appearances you have,
stopping once you have a number with appearances $> \lceil \frac{n}{2} \rceil$.

(b) What if we want to know if there are at least $\lceil \frac{n}{100} \rceil$ numbers with equal value? Justify correctness.

It would be the same method as the one above, except the second loop would end earlier, when you found an item with appearances $> \lceil \frac{n}{100} \rceil$

Its $O(n)$ because we iterate over the array twice, and $2n = O(n)$.

\pagebreak

\item
We release $k$ bees in a field with $n$ flowers.  $k$ might be smaller, equal, or larger than $n$.
Each bee decides to go to some random flower.  Multiple bees can land on the same flower.

(a) What is the expected number of bees that will visit each flower?

$\frac{k}{n}$

(b) How many flowers do we expect will be visited?

Up to $k$ flowers, but not more than $n$ flowers.

Using Indicator Random Variables, we can use the expression

Every extra bee we add to the equation has a $\frac{n - i}{n}$ chance of landing on a fresh flower,
where $i$ is the number of bees that have currently landed on flowers.

$$\sum_{i=1}^{k} \frac{n - i}{n}$$
to show the likelihood of a bee landing on each flower.


(c) Does your solution for (b) confirm the  intuitive answer for the special case where there is only one flower?  Or what if there's only one bee?   
 If we release 400 bees and there are 100 flowers, what's the answer?\\

It is slightly inaccurate for the case of a single flower.

For a single bee this might also be slightly inaccurate.

If $k=400$ and $n=100$, we expect 100 flowers to be visited.



\end{enumerate}



\end{document}






















